\chapter{Introduction}

System calls are the standard way through which user programs request services
from the operating system. Every time an application needs to write to disk,
send a network packet or obtain system information it needs to ask the
operating system to do it on its behalf. While this segregation of duties is
essential for security reasons, it also incurs performance overhead. This
overhead can vary depending on different factors such as the underlying
architecture, the system configuration, and whether or not the operating system
runs in a virtualized environment.

A great deal of research for finding ways to minimize system call related
overhead has been done already over the years
\cite{clustering}\cite{compositecalls}\cite{cassyopia}\cite{compositecalls2}.
Even though most research has been focused on system call batching techniques,
a previous study \cite{dedicated_cpus} from the Portland State University
claims to have achieved performance improvements by using dedicated user and
kernel CPUs. This study however was not performed in the context of a
virtualized operating system.

Recently, for different reasons, the para-virtualized \llinux kernel has
introduced an execution mode called ``decoupling" \cite{decoupling}. This mode
allows user threads to run isolated on dedicated CPU cores others than the ones
used for executing the \llinux kernel. In this case, it has resulted in reduced
system call performance for decoupled applications. Finding ways to minimize
this additional overhead is essential for allowing applications issuing large
amounts of system calls to run as decoupled threads.

\section{Thesis Objectives}

The main objective of this thesis is to design and implement a forwarding
mechanism that allows user threads bound to a certain CPU core to invoke system
calls on a Linux kernel running on a different core. The new method must be
faster than the slow existing mechanism. Existing Linux programs should be able
to use the new mechanism without requiring any modification to their source
code.

A secondary objective is to investigate the potential performance improvements
achieved using an asynchronous system call execution model and provide new
software applications with a way to use it.

\section{Document Structure}

This document is organized as follows. Chapter \ref{chap:fundamentals} provides
the reader with the necessary background for understanding the following
chapters and discusses related work. Chapter \ref{chap:design} introduces the
new forwarding mechanism and presents its overall architecture and main design
points. Chapter \ref{chap:implementation} discusses a few interesting
implementation aspects of the new mechanism. Chapter \ref{chap:evaluation}
presents the results obtained through the experimental evaluation of this work.
Finally, Chapter \ref{chap:conclusion} concludes the thesis and proposes future
work directions.
