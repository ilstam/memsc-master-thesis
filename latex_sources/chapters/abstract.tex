\chapter*{\centering Abstract}

The L4Re microkernel has long been able to function as a hypervisor and host a
slightly modified Linux kernel version called \llinux. Recently, \llinux
introduced a new mode of operation which allows application threads of Linux
programs to decouple themselves from the Linux scheduler. These threads can
then run uninterrupted on dedicated CPU cores managed by the L4Re microkernel
while \llinux keeps running on a separate core. When decoupled threads need to
request Linux services using system calls, these requests have to be forwarded
from the CPU cores hosting the decoupled threads to the one hosting \llinux.
This forwarding requires expensive cross-processor communication which involves
the microkernel and uses inter-processor interrupts that cause high latency.

This thesis introduces \memsc, a new system call forwarding mechanism
implemented in the \llinux kernel. Memsc works by having \llinux perform memory
polling on a set of special shared memory pages that registered processes use
for posting system call entries, instead of passing system call arguments via
CPU registers. Involvement of the L4Re microkernel is not required at any step
in this process. With the help of a modified C library, applications can use
\memsc transparently without requiring modification or recompilation. New
applications willing to use the user-space API directly can further benefit
from additional capabilities such as an asynchronous system call execution
model. The experimental evaluation of \memsc, conducted using both custom
microbenchmarks and general-purpose file system benchmarks, shows that it
performs significantly better than the existing \llinux system call forwarding
mechanism.
